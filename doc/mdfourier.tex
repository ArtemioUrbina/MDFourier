\documentclass[10pt,a4paper]{report}
\usepackage[utf8]{inputenc}
\usepackage[T1]{fontenc}
\usepackage{graphicx}
\usepackage{amsmath}
\usepackage{float}
\usepackage{xfrac}
\usepackage[pdfpagelabels]{hyperref}
\setlength{\parskip}{1em}

%opening
\title{MDFourier}
\author{Artemio Urbina}

\begin{document}

\pagenumbering{Alph}
\begin{titlepage}
	\maketitle
	\thispagestyle{empty}
\end{titlepage}
\pagenumbering{arabic}

\tableofcontents

\chapter{MDFourier Objective}

Provide a Fourier based analysis framework to compare audio generated by any Mega Drive/Genesis variant, clone or implementation. This is of course not limited to the Mega Drive/Genesis, and the software with minor changes can be used to compare any other platform.

The intention is not to disparage any particular implementation, but help to understand and improve them whenever possible by identifying the differences. This can help emulators, \textit{FPGA} implementations and even hardware modifications to better match the desired reference recording.

\section{What is it for?}

It can be used to identify the differences between two audio signatures. These can come from iterations of the console family, like a Sega Genesis Model 1 VA3 and a Sega Model 2 VA 1.8, or even between a console and a hardware modification, recapping the system, an emulator, or an FPGA implementation.

These results can be used to determine if the signals are indeed different, and how they differ across the human hearing frequency spectrum. Such information can then be used for different means, such as recreating specific audio signatures, tuning to a different taste, etc; based on an objective, repeatable and measurable data set and framework.

Although I believe any present and future implementation should offer a configuration based on vintage retail hardware, that doesn't mean there isn't room for improvement upon those configurations. Reducing noise while keeping the original sound signature is one such case.

\section{Warning}

MDFourier is a work in progress, just as this document. It still has a few rough edges, and although I tried to adhere to the best practices I know, I am no expert in the matters of \textit{digital signal processing}. If you have any suggestions please contact me \ref{contact} to help improve this document and the software in general.

\chapter{Requirements}

\section{Audio capture device}
For capturing the audio files, an audio capture device is needed. It is recommended to use a musical grade audio card in order to get a flat frequency response across the human hearing spectrum.

So far two audio cards have been used in my personal setup. The reference recordings available for download were made with both cards, a set with each one of them.

The author of this software has no association or business relationship with these products, they are just presented as the references used. As more people use the software and we can compare files, we can expand this list with recommendations.

\begin{itemize}
	\item \textbf{M-Audio Audiophile 192:} An internal audio card that is no longer available in the market \cite{maudio}
	\item \textbf{Lexicon Alpha:} An affordable USB audio card. \cite{lexicon}
\end{itemize}

We have tested some cards that don't have a flat frequency response, you should try to use a sound card that is aimed to musicians or instrument recording.

\section{Computer}

Any computer can be used if you are compiling the source code from scratch \cite{sourcecode}, but a statically linked \textit{Microsoft Windows executable} is provided for convinience, alongside a front end. See chapter \ref{usinggui} for instructions on using the GUI.

\section{Game Consoles or emulators}

You'll need either the provided example audio files or create your own by recording from the desired source, which makes more sense since you probably want to compare how these behave.

\section{Flash cart, or means to run the binary}

The console needs to run a custom built binary, a ROM. I will build this functionality into each version of the \textit{240 test suite}\cite{240pSuite} as possible.

In order to run these, you'll either need a flash cart or a custom loading solution compatible with the target platform.

\section{Cables and adapters}

You'll need cables and maybe some adapters to connect the audio output fomr the console to the input of your audio capture card.

\section{Audio capture software}

Your capture card will probably be bundled with some audio editing software, or you can use  Audacity\cite{audacity} or Goldwave\cite{goldwave} options depending on your operating system.

\chapter{How MDFourier works}
\label{howitworks}

The first thing to keep in mind is what \textit{MDFourier} does. It takes two signals, the first one is the \textit{Reference} file and the second one is the \textit{Comparison} file.

The \textit{Reference} file is used as a control. This means that its characteristics are considered the true values to be expected and against which the \textit{Comparison} file will be evaluated.

These files are audio recordings from the desired hardware, captured with a flat frequency audio capture card and generated by a custom binary. This binary will be part of the \textit{240p Test Suite}\cite{240pSuite} when possible, and a custom binary for each targeted platform that covers the hardware capabilities and frequency range.

The software is itself command line based, in order to be multi platform and offer it on every operating system that has an \textit{ANSI C99 compiler}. However a GUI front end is provided for simplicity and accessibility. Not all options from the command line tool are currently available via the GUI, but most relevant ones are readily available. Full Source code is available under the \textit{GNU GPL} at Github \cite{sourcecode}.

\section{File alignment}

\textit{MDFourier} takes both files and auto detects the starting and ending point of the recording. These are identified by a series of 8020hz pulses in the current \textit{Mega Drive/Genesis} implementation. From these, a frame rate is calculated in order to trim each file into the segments that are defined in the configuration file\ref{mfnconfig} for further comparison.

Most importantly, it guarantees that the \textit{Reference} and \textit{Comparison} files are logically aligned, and that each note or segment is compared to the corresponding one, with no overlap and without any skills required on audio editing or trimming from the user. Current accuracy is $\sfrac{1}{4}$ of a millisecond.

After alignment is accomplished, the software reports both starting and end points of both signals, in seconds and bytes. A specialized tool called MDWave is included, that can trim each file and segment each block for acoustical and visual verification if required. At the moment MDWave has no GUI Front end available.

\section{Configuration file}
\label{mfnconfig}

All these parameters are defined in the file \textit{mdfblocks.mfn}. Here is the current file we are using to compare \textit{Mega Drive/Genesis} audio characteristics:

\begin{verbatim}
              MDFourierAudioBlockFile 1.0
              MegaDriveAudio
              16.6905
              8820 -25 25 14 18 10
              7
              Sync s 1 20 red
              Silence n 1 20 red
              FM 1 96 20 green
              PSG 2 60 20 yellow
              Noise 3 14 20 aqua
              Silence n 1 20 red
              Sync s 1 20 red
\end{verbatim}

This file defines what \textit{MDFourier} must do and how to interpret the WAV files. For now it can read \textit{44khz} and \textit{48khz} files, in \textit{Stereo PCM} format.

The first line is just a header, so that the program knows it is a valid file and in the current format.

The second line is the name of the current configuration, since I plan to add support for any console or arcade hardware in the future. This would imply creating a new \textit{mfn} file for each configuration, and a specific binary to be run on the hardware.

The third line is the expected frame rate. This is only used as a reference to estimate the placement for the blocks within the file before calculating the frame rate as captured by the audio capture card. After that is calculated, each file uses its own definition in order to be fully aligned. Variations in the detected frame rate are natural, since we have an error of $\sfrac{1}{4}$ of a millisecond, dictated by the sample rates and audio card limitations. For reference $\sfrac{1}{4}$ of a millisecond corresponds to 0.000125 seconds.

The third line defines the characteristics of the pulse tone used to identify the starting and end points of the signal within the wave file. Its frequency, relative amplitude difference to the background noise (silence), and length intervals that will be better explained on a future revision of this document.

The fourth line defines how many different blocks are to be identified within the files. There are seven blocks in this case.

Each block is composed of five characteristics: A Name, a \textit{type}, the \textit{total number of elements} that compose it, each element \textit{duration} specified in frames and the \textit{color} to be used for identifying it when plotting the results.

For example, FM audio has been named \textit{"FM"}, type \textit{1}, \textit{96} elements of \textit{20} frames each and will be colored in \textit{green}. Definition is in frames since emulators and \textit{FPGA} implementations tend to run at different frame rates than the original platform, which result in different durations. The only way to align them, is by respecting the driving force in hardware like these: the video signal.

There are currently two special types, identified by the letters \textit{'s'} and \textit{'n'}. The first one defines a \textit{sync pulse}, which is used to automatically recognize the start and ending points of the signal within the wave file. 

The second one is for null audio, or silence. This \textit{silence} is used to measure the background noise as recorded by the audio card. 

\section{The heart of the process}

In order to compare the signals, audio levels are checked and a relative normalization takes place, based on the maximum amplitude of the \textit{Reference} file. A local maximum search is done in the same frame on the \textit{Comparison} signal, and that amplitude is then used to normalize both files. This is done in the time domain by default, in order to reduce amplitude imprecision caused when comparing recordings with different frame rates. The software does have the option to do this in the frequency domain, but quantization was deemed to be negligible in my results.

Then a \textit{Dicrete Fourier Transform} is used in order to analyze the frequency content of the signal, as well as the amplitudes from each of the corresponding fundamental frequencies that compose the signal.

The software uses the \textit{FFTW}\cite{fftw} library in order to accomplish this, and then proceeds to sort out the frequencies of each block by amplitude. It can be configured to compare a range of these frequencies, but by default it compares 2000 of them for each element defined in the \textit{mfn} configuration file (see section \ref{mfnconfig}).

I've found such comparison to be more than enough, and the minimun significant volume (section \ref{MinSigVolume}) even limits these 2000 to a lower number based on significant amplitude. In case more frequencies are needed, this can be changed via the command line parameters.

After comparing these frequencies between both files, matches are made and the differences in volume are plotted to a graph. Please read the corresponding chapter to help you understand the results. (see chapter \ref{howtoplots})

Sometimes it is helpful to listen to the results of these filterings, in order to know if 2000 frequencies are enough or too little for the current scenario. For this and other purposes I created an extra tool named \textit{MDWave}\ref{mdwave}, which creates the segmented wave files as processed with the Fourier transform, window filters included (section \ref{windows}).

If you are interested in learning what the Fourier Transform does and how it works it's magic, there are several resources online to help you out. Here are a few:

\begin{itemize}
	\item But what is the Fourier Transform? A visual introduction.  \url{https://www.youtube.com/watch?v=spUNpyF58BY}
	\item The Uncertainty Principle and Waves - Sixty Symbols.  \url{https://www.youtube.com/watch?v=VwGyqJMPmvE}
	\item An Interactive Introduction to Fourier Transforms.  \url{http://www.jezzamon.com/fourier/index.html}
\end{itemize}


\section{Minimal significant volume}
\label{MinSigVolume}

Currently \textit{MDFourier} can recognize three scenarios used as a minimum significant volume to compare the signals, and the three are derived from the first \textit{silence block} in the file.

The first scenario is the grid power frequency noise, which currently searches for \textit{60/50hz} noise (\textit{PAL} needs to be tested yet, since I don't have a PAL console and its use is triggered based on the identified frame rate). The second one is refresh rate noise, again derived from the frame rate, and around \textit{15697-15698hz}. 

In case neither is found, which would be surprising for a file generated by recording from a real console via analogue means, the frequency with the highest volume within the silence block is used. 

In case neither of the three scenarios is met, a default $-96bDBFs$ level is used.


\section{Workflow}

The first step is loading the custom binary to your console, this varies from a flash cart, burning a CD or using a custom loader.

The next step is seting up your audio card and computer, in order to record a \textit{44khz 16 bit stereo} wav file.

Once the capture card is ready and the cables are hooked up from the console to the capture card, start recording on the computer and start the \textit{MDFourier} test from the console.

Wait for the console to show a message indicating you can stop recording, this typically takes less than 1 minute. After you have at least two files, a reference which can be one of the provided files, and a comparison file you are ready to go. 

\chapter{How to use the Front End}
\label{usinggui}
The current version of the front end allows access to the main options of \textit{MDFourier}. It is a Windows executable and all corresponding files must be placed in the same folder. Uncompressing the package to a folder should have all that is necessary to run the programs.

After running \textit{MDFourierGUI.exe} you should be presented with the following interface:

\begin{figure}[H]
	\centering
	\includegraphics[width=0.6\linewidth]{plots/GUI1.png}
	\caption[Front End]{MDFourier Windows Front End}
	\label{fig:gui1}
\end{figure}

In order to generate the output plots, two files must be selected to compare them. One as a \textit{Reference} and the other as the \textit{Comparison} file as detailed in section \ref{howitworks}.

The following sequence of steps indicates the typical work flow within the GUI:

\begin{figure}[H]
	\centering
	\includegraphics[width=0.6\linewidth]{plots/GUI2.png}
	\caption[Steps]{Typical sequence of steps}
	\label{fig:gui2}
\end{figure}

\begin{enumerate}
	\item Select a \textit{Reference} file
	\item Select a \textit{Comparison} file
	\item Change the default \textit{options} if needed
	\item Execute \textit{MDFourier}
	\item When execution ends, open the \textit{results folder}
\end{enumerate}

The Front End will display the output text from the command line tool, including any errors or progress available.

Keep in mind that the \textit{Open results} button will only be enabled after a successful comparison between files has been finished, and it won't open a second instance of the window if you have one already open.

\section{Front End Options}

The currently available options in the Front End are:

\begin{itemize}
	\item \textbf{DFFT Window:} In order to reduce spectral leakage a filtering window is applied to each element compared between both signals. Please consult section \ref{windows} for details. 
	\item \textbf{Color Filter Function:} This is a \textit{filter function} applied to the results in order to highlight or attenuate the differences between the files.  Please consult section \ref{colorfilter} for details. 
	\item \textbf{Align FFTW to 1hz:} Creates a version of each trimmed note, zero padded to match the sample rate in order to align the FFT bins to 1hz, as a  result there will be more dots plotted. Off by default.
	\item \textbf{Average Plot:} This traces an average line on top of the plots, making it easier to follow the data trend when the comparison file has severley scattered data. Off by default.
	\item \textbf{Verbose Log:} This option creates a detailed log in the oputput folder, useful for reporting errors or unexpected behaviour. (\textit{Please send the wav files if possible as well!})
\end{itemize}

\section{Window Functions}
\label{windows}

In order to reduce \textit{spectral leakage} a filtering window is applied to each element compared between both signals. Since we are generating the signal ourselves from the custom binary for each hardware platform, the signal can be analyzed as periodic and has a natural attack and decay rate.

By default we use a custom \textit{Tukey window} with very steep slopes. \textit{MDFourier} does offer alternate windows as options for further analysis. You can read more about windows and their usage in the reference web page\cite{windowtypes}.

\subsection{Tukey}

The default is a Tukey window specifically designed for this purpose. It uses a 2.5\% slope on each side of the signal, zeroing just a few samples and with minimal amplitude and frequency distortion.

The following equation is used to create the slopes:

\begin{equation}
tukey(x)=85(1+\cos(\frac{2\pi}{n-1}\frac{x-(n-1)}{2}))
\end{equation}

And this is the resulting plot of the Tukey window, tanges are 0-1 horizontally and -0.1 to 1.1 vertically.

\begin{figure}[H]
	\centering
	\includegraphics[width=0.4\linewidth]{plots/window-tukey.png}
	\caption[Tukey Window]{This is the custom Tukey window used by MDFourier}
	\label{fig:window-tukey}
\end{figure}


\subsection{Flattop}
A typical Flat top window is used.

\begin{align*}
flattop(x)=0.21557895 - 0.41663158\cos(2\pi\frac{x}{n-1})+ 0.277263158\cos(4\pi\frac{x}{n-1})\\
 - 0.083578947\cos(6\pi\frac{x}{n-1}) + 0.006947368\cos(8\pi\frac{x}{n-1})
\end{align*}

\begin{figure}[H]
	\centering
	\includegraphics[width=0.4\linewidth]{plots/window-flattop.png}
	\caption[Flat Top window]{Flat Top window}
	\label{fig:window-flattop}
\end{figure}


\subsection{Hann}
A typical Hann window is used.

\begin{equation}
hann[x] = \frac{1}{2}(1 - \cos(\frac{2\pi(x+1)}{n+1}))
\end{equation}

\begin{figure}[H]
	\centering
	\includegraphics[width=0.4\linewidth]{plots/window-hann.png}
	\caption[Hann Window]{Hann Window}
	\label{fig:window-hann}
\end{figure}


\subsection{Hamming}
A typical Hamming window is used.

\begin{equation}
hamming[x] = 0.54 - 0.46\cos(\frac{2\pi x}{n-1})
\end{equation}

\begin{figure}[H]
	\centering
	\includegraphics[width=0.4\linewidth]{plots/window-hamming.png}
	\caption[Hamming window]{Hamming window}
	\label{fig:window-hamming}
\end{figure}


\subsection{No Window}

No window is applied, equivalent to a rectangular window. This leaves the signal unprocessed and any uncontrolled decay and audio card noise will be factored in as part of the periodic signal.


\section{Color Filter Functions}
\label{colorfilter}

Since each dot in the \textit{Difference} graph uses the X axis for the frequency range and the Y axis for the amplitude difference the \textit{Comparison} file has against the \textit{Reference} file, the colr intensity is used to represent how high the volume for that frequency was in the \textit{Reference} file. 

A color scale is presented in each graph, with teh color graduation and the corresponding volume level.

This color represents the amplitude of the sine wave from the \textit{Reference} signal. In other words, how relevant it was to create the original signal in that note. Please refer to chapter \ref{howtoplots} in order to see examples of their use.

The options are useful to highlight or attenuate these difefernces by applying the range to one fo the following functions. They are sorted in descending order for highlighting differences, from showing all of them to just the ones with the highest amplitude.

\subsection{None} 

No filtering is applied, as a result all differences are plotted with the brightest color. 

\begin{figure}[H]
	\centering
	\includegraphics[width=0.4\linewidth]{plots/BetaFunctionPlot_0}
	\caption[No Filter]{No Filter}
	\label{fig:betafunctionplot0}
\end{figure}

\begin{figure}[H]
	\centering
	\includegraphics[width=1\linewidth]{plots/BetaFunctionPlot_0_Data}
	\caption[No Filter]{No Filter Applied}
	\label{fig:betafunctionplot0data}
\end{figure}

\newpage
\subsection{$\sqrt{dbFS}$} 

A square root function will only attenuate the lowest amplitude differences.

\begin{figure}[H]
	\centering
	\includegraphics[width=0.4\linewidth]{plots/BetaFunctionPlot_1}
	\caption[Square Root filter]{Square Root filter}
	\label{fig:betafunctionplot1}
\end{figure}

\begin{figure}[H]
	\centering
	\includegraphics[width=1\linewidth]{plots/BetaFunctionPlot_1_Data}
	\caption[Square Root filter]{Square Root filter Applied}
	\label{fig:betafunctionplot1data}
\end{figure}

\newpage
\subsection{$\beta(3,3)$} A beta function is applied

A Beta Function filter with parameters 3, 3 will attenuate a bit more from the lower range, still showing most of the differences.

\begin{figure}[H]
	\centering
	\includegraphics[width=0.4\linewidth]{plots/BetaFunctionPlot_2}
	\caption[Beta Function(3,3)]{Beta Function(3,3)}
	\label{fig:betafunctionplot2}
\end{figure}

\begin{figure}[H]
	\centering
	\includegraphics[width=1\linewidth]{plots/BetaFunctionPlot_2_Data}
	\caption[Beta Function(3,3)]{Beta Function(3,3) Applied}
	\label{fig:betafunctionplot2data}
\end{figure}

\newpage
\subsection{$Linear$} 

The linear function is the default, and has no bias. Half the dinamic range corresponds to half the color rage.

\begin{figure}[H]
	\centering
	\includegraphics[width=0.4\linewidth]{plots/BetaFunctionPlot_3}
	\caption[Linear]{Linear Function}
	\label{fig:betafunctionplot3}
\end{figure}

\begin{figure}[H]
	\centering
	\includegraphics[width=1\linewidth]{plots/BetaFunctionPlot_3_Data}
	\caption[Linear Applied]{Linear Function Applied}
	\label{fig:betafunctionplot3data}
\end{figure}

\newpage
\subsection{$dBFS^2$}

A squared function will attenuate a lot more differences, and most frequencies with the highest amplitude in the reference signal will be brighter.

\begin{figure}[H]
	\centering
	\includegraphics[width=0.4\linewidth]{plots/BetaFunctionPlot_4}
	\caption[Linear]{Linear Function}
	\label{fig:betafunctionplot4}
\end{figure}

\begin{figure}[H]
	\centering
	\includegraphics[width=1\linewidth]{plots/BetaFunctionPlot_4_Data}
	\caption[Linear Applied]{Linear Function Applied}
	\label{fig:betafunctionplot4data}
\end{figure}

\newpage
\subsection{$\beta(16,2)$} 

A Beta Function filter with parameters 16,2 will attenuate almost all the differences, and only the frequencies with the highest amplitude in the reference signal will be brighter.

\begin{figure}[H]
	\centering
	\includegraphics[width=0.4\linewidth]{plots/BetaFunctionPlot_5}
	\caption[Beta Function(16,2)]{Beta Function(16,2)}
	\label{fig:betafunctionplot5}
\end{figure}

\begin{figure}[H]
	\centering
	\includegraphics[width=1\linewidth]{plots/BetaFunctionPlot_5_Data}
	\caption[Beta Function(16,2)]{Beta Function(16,2) Applied}
	\label{fig:betafunctionplot5data}
\end{figure}

\chapter{How to understand the plots}
\label{howtoplots}

The main output of the program is a set of different graphics that vary in quantity based on the definitions made in the \textit{mfn}\ref{mfnconfig} file detailed in the previous section.

In its current form for the \textit{Mega Drive/Genesis}, there are three \textit{active blocks}: \textit{FM}, \textit{PSG} and \textit{Noise}. These will result in a plot of each type, and a general plot, being generated as output.

The files are saved under the folder \textit{MDFourier} and a sub-folder named after the input \textit{WAV} file names. They are stored in PNG format, currently 1600x800 plots are used, although this can be dynamic. For the current document 800x400 plots were used in order to fit within a PDF or HTML presentation.

\section{Output Files}
There are common features here that we'll describe. First the type of output files: 

\begin{itemize}
	\item \textbf{DifferentAmplitude:} Plots the amplitude difference for the frequencies common to both files
	\item \textbf{MissingFrequencies:} Plots the frequencies available in the \textit{Reference} file but not found in the \textit{Comparison} file within the significant volume range.
	\item \textbf{Spectrogram:} Plots all the frequencies available in each file. Two sets of spectrograms are generated, one for the \textit{Reference} file and one for the \textit{Comparison} file.
\end{itemize}

As mentioned above, there will be several plots of each type in the output folder. One for each type, and one that summarizes all types in a single plot.

In our current\textit{ Mega Drive/Genesis} scenario, we'll get four plots of each type: \textit{FM}, \textit{PSG}, \textit{Noise} and a general one, named \textit{ALL}.

We'll follow a series of results from differen input files to MDFourier, starting with cases that have either none or a few differences, and build on top of each one so you can familiarize with what to expect as output.

\section{Scenario 1: Comparing the same file against itself}

The first scenario we'll cover is the basic one, the same file against itself. Let's keep in mind that \textit{MDFourier} is designed to show the relative differences between two audio files.

So, what is the expected result of comparing a file to itself? No differences at all, an empty plot file as shown below.

\begin{figure}[H]
	\centering
	\includegraphics[width=1.0\linewidth]{plots/Plot1-SameFile.png}
	\caption[Same file compared]{Different Amplitudes result file when comparing the same file against itself.}
	\label{fig:plot1-samefile}
\end{figure}

Of course all of the \textit{Differences} and \textit{Missing} plots will only have the grid and reference bars, with no plotted information since both input files are identical. 

However there will be two sets of \textit{Spectrograms}, one for the \textit{Reference} file and one for the \textit{Comparison} file, with one plot for each type plus the general one.

\begin{figure}[H]
	\centering
	\includegraphics[width=1.0\linewidth]{plots/Plot2-SameFile-FM-Spectrogram.png}
	\caption[Spectrogram]{The Spectrogram for a Genesis 1 VA3 via hedphone out}
	\label{fig:plot2-samefile-fm-spectrogram}
\end{figure}

The Amplitude, or volume, of each of the fundamental sine waves that compose the original signal is represented by vertical lines that reach from the bottom to the point that represents the amplitude in \textit{dBFS}. The line is also colored to represent that amplitude with the scale on the left showing the equivalence.

Three colors as defined from the \textit{mfn file} are used to plot the graph, with each one of them plotting the frequencies from each corresponding block from the \textit{WAV} file.

The top of the plot corresponds to the maximum possible amplitude, which is \textit{0 dBFS}. the bottom of the plot corresponds to the \textit{minimum significant volume}, as described in section \ref*{MinSigVolume}.

Both sets of spectrograms in this case are identical as expected. 

\section{Scenario 2: Comparing two different recordings from the same console}

This is another control case, what shoudl we expect to see if we record two consecutive audio files from the same exact game console using the same sound card?

\begin{figure}[H]
	\centering
	\includegraphics[width=1\linewidth]{plots/Plot2-Sameconsole}
	\caption{}
	\label{fig:plot2-sameconsole}
\end{figure}

As you can see, we have basically a flat line around zero. This means that there were no meaningful differences found. 

But wait, there are differences. Why is that? Due to many reasons: analogue recordings are not always the same for one. Then we have variations from the analogue part of console itself, and probably from teh internal states and clocks from the digital side. It can also be noise generated by differences in frequency bins when performing the FFTW after calculationg the frame rates, we have that $\sfrac{1}{4}$ error after all.

We now know that there will be certain fuzziness or variation around each plot due to this subtle recording and perfmance variations. It is a normal situation that is to be expected, and a baseline for future results.

\chapter{MDWave}
\label{mdwave}

\textit{MDWave} is a companion command line tool to \textit{MDFourier}. While developing the tool suite and learning about \textit{DSP}, I needed to check what I was doing and visualise or listen to the results. Hence, \textit{MDWave} was born.

It takes a single wave file as argument, and follows all the parameters defined in the configuration file (see section \ref{mfnconfig}).

It has a few more commmand line options, which I'll detail in later versions of the document. You can type \textit{mdwave -h} in your mdfourier folder for details.

\chapter{Contact the author}
\label{contact}

You can contact me via twitter \url{http://twitter.com/Artemio} or e-mail me at aurbina@junkerhq.net

\begin{thebibliography}{9}
	\bibitem{sourcecode}
	Github,
	\textit{MDFourier source code C99},
	\url{https://github.com/ArtemioUrbina/MDFourier}.
	
	\bibitem{maudio}
	M-Audio Audiophile 192,
	\textit{Specifications},
	\url{https://www.soundonsound.com/reviews/m-audio-audiophile-192}.
	
	\bibitem{lexicon}
	Lexicon Alpha USB card,
	\textit{Product web page},
	\url{https://lexiconpro.com/en/products/alpha}.
	
	\bibitem{240pSuite}
	240p test Suite,
	\textit{Wiki web page},
	\url{http://junkerhq.net/240p}.
	
	\bibitem{audacity}
	Audacity
	\textit{web page},
	\url{https://www.audacityteam.org/}.
	
	\bibitem{goldwave}
	Goldwave
	\textit{product web page},
	\url{https://www.goldwave.com/}.
	
	\bibitem{fftw}
	Fastest Fourier Transform in the West.,
	\textit{web page},
	\url{http://fftw.org/}.
	
	\bibitem{windowtypes}
	Window Types: Hanning, Flattop, Uniform, Tukey, and Exponential ,
	\textit{web page},
	\url{https://community.plm.automation.siemens.com/t5/Testing-Knowledge-Base/Window-Types-Hanning-Flattop-Uniform-Tukey-and-Exponential/ta-p/445063}.
\end{thebibliography}

\end{document}
